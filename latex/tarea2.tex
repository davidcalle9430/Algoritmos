1 Escribir Formalmente  el algoritmo de búsqueda lineal

procedure linear_search ( L, V ) do
  FOR i <- 1 to | L | do
    IF L[ i ] = V do
      RETURN i
  RETURN N
1.1 Demostrar que el algoritmo es correcto  en terminos de las invariantes de ciclo
Hecho

Entradas: Una secuencia L  de n números S = { a1, a2, a3, ... , an }  y un valor v.
Salidas: Un indice i  // copiar resto de expresión en LATEX

Invariante

Todo número por el que ha pasado no es el número que se está buscando

Inicialización:
  i = 0, es menor que | L |
Mantenimiento:
  Al inicio de cada ciclo para todo si j < i -> S[i]  != v  
Terminación:
  J = N



procedure count_repeated ( L ) do
  SET return_set
  FOR i <- 1 to | L | do
    add ( return_set )
  RETURN | L | - | return_set |
invariantes

inicialización
  el set inicial unido con el mismo es igual porque no tiene nigún elemento
mantenimiento
   i > 0 & i < | L |, set_return  unión set_return = set_return
terminación
  no se
